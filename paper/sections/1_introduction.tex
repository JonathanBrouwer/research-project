\section{Introduction}
Haskell is a strongly typed purely functional programming language \cite{haskell}. A big advantage of this is that it makes reasoning about the correctness of algorithms and data structures relatively simple. However, these proofs are all done on paper, and making a mistake in these proofs is notoriously easy. There is also always the risk that the proof is no longer valid after the code changes. Agda is a dependently typed programming language and interactive theorem prover \cite{agda}.  Using Agda and the Curry-Howard correspondence \cite{chc}, one can write a formal proof about the code in the language itself, and use the compiler to verify the correctness of the proof \cite{schwaab, van}. The compiler also verifies that the proof is still valid each time the code changes.

The Agda2hs \cite{agda2hs} is a project that identifies a common subset of Agda and Haskell, and provides a tool that automatically translates code from this subset of Agda to Haskell. This makes it possible to write the program in this subset, using full Agda to prove properties about it, and then translate it to nice looking readable Haskell code. However, agda2hs is not finished yet, as it still lacks some Agda features that it cannot compile to Haskell. It is also not yet known how much effort extra it takes to write code in this subset of Agda.

In this paper, the QuadTree library is implemented and verified in this subset of Agda to determine whether agda2hs can be used to produce a verified implementation of a Haskell library (section 3.1). If this turns out to be difficult, changes to agda2hs or the library to make this possible will be determined (section 3.2-3.3). Then invariants, preconditions, and postconditions of the library will be stated and the techniques used to prove them will be shown shown (section 4). Section 5 discusses responsible research. Finally, the results are discussed (section 6) and the paper is concluded (section 7).
