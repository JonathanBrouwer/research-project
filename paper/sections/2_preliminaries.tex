\section{Preliminaries}
\subsection{QuadTrees}
The QuadTree is a data structure that is used for storing two-dimensional information in a functional way. \cite{Finkel1974}. It is defined as:
\begin{minted}{haskell}
data Quadrant t = Leaf t
        | Node (Quadrant t) (Quadrant t) (Quadrant t) (Quadrant t)

data QuadTree t = Wrapper (Nat, Nat) (Quadrant t)
\end{minted}

\begin{wrapfigure}{r}{0.3\textwidth} %this figure will be at the right
    \includegraphics[width=0.3\textwidth]{graphics/test.png}
    \caption{An example QuadTree}
\end{wrapfigure}

A QuadTree consists of the size of the QuadTree, and the root quadrant. A quadrant is either a leaf (in which case all the values inside the region of the quadrant is the same), or four other nodes. The four subquadrants are then called A (top left), B (top right), C (bottom left), and D (bottom right). Notice that in Figure 1, space is consistently split into four quadrants.

There are five functions that can be used to interact with QuadTrees:
\begin{minted}{haskell}
-- Create a new QuadTree with the specified size
makeTree :: (Nat, Nat) -> t -> QuadTree t
-- Get a lens to the specified location
atLocation :: (Nat, Nat) -> Lens (QuadTree t) t
-- Get the value at the specified location
getLocation :: (Nat, Nat) -> QuadTree t -> t
-- Set the value at the specified location
setLocation :: (Nat, Nat) -> t -> QuadTree t -> QuadTree t
-- Map the value at the specified location
mapLocation :: (Nat, Nat) -> (t -> t) -> QuadTree t -> QuadTree t
\end{minted}

\subsection{Lenses}
The QuadTree library makes extensive use of Lenses. Lenses are composable functional references \cite{lens}. They allow one to access and modify data in a data structure. This paper chooses to use the Van Laarhoven representation \cite{laarhovenlens}, since this is what the original library used. It is defined as:
\begin{minted}{haskell}
type Lens s a = forall f. Functor f => (a -> f a) -> s -> f s
\end{minted}
And the functions to interact with Lenses are:
\begin{minted}{haskell}
-- Get the value at this lens
view :: Lens a b -> a -> b
-- Set the value at this lens
set :: Lens a b -> b -> a -> a
-- Map the value at this lens
over :: Lens a b -> (b -> b) -> a -> a
-- Compose two lenses (Note: This is actually just regular function composition!)
compose :: Lens a b -> Lens b c -> Lens a c
\end{minted}