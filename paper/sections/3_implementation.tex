\section{Implementation} \label{implementation}
This section describes how the QuadTree library was implemented in Agda, and what challenges had to be overcome to do so. All the code for this project is available in the public domain. Each directory has a README.md which explains the purpose of all files and folders inside of the directory.
It is available at: \textcolor{blue}{\href{https://github.com/JonathanBrouwer/research-project}{https://github.com/JonathanBrouwer/research-project}}. 

\subsection{Implementing QuadTree}
The QuadTree library is implemented by composing lenses. The lens that is finally produced is the atLocation lens, which takes a location and a QuadTree, and returns a lens to that location in the QuadTree.
\begin{minted}{agda}
atLocation : (Nat x Nat) -> Lens (QuadTree t) t
\end{minted}
 \verb|atLocation| is implemented by composing \verb|wrappedTree| (which is a lens from a QuadTree to its root quadrant) and  \verb|go|.   The \verb|go| function takes a location and a maximum depth, and returns a lens from a quadrant to the specified location. In its implementation, if the maximum depth is zero, it calls  \verb|lensLeaf|. Otherwise, it composes  \verb|lensA/B/C/D| with a recursive call to itself, that does the rest of the lookup. For example, \verb|go (0 , 0) 5 = lensA ∘ go (0 , 0) 4|.
\begin{minted}{agda}
lensWrappedTree : Lens (QuadTree t) (Quadrant t)
go : (Nat x Nat) -> (depth : Nat) -> Lens (Quadrant t) t
\end{minted}
 \verb|lensLeaf| is a lens from a leaf quadrant to the value stored there. This function has as a precondition that the quadrant has a depth of 0 (a leaf).  \verb|lensA/B/C/D| is a lens from a quadrant to the A/B/C/D sub-quadrant. This function returns a lens from a quadrant with a certain maximum depth to a quadrant with a maximum depth that is one lower. 
\begin{minted}{agda}
lensLeaf : Lens (Quadrant t) t
lensA : Lens (Quadrant t) (Quadrant t)
\end{minted}
 \verb|get/set/mapLocation| can then be defined using the  \verb|atLocation| lens, by composing them with the lens functions. They are shortcut functions so users of the library do not have to interact with lenses directly.
\begin{minted}{agda}
getLocation : (Nat x Nat) -> QuadTree t -> t
getLocation = view ∘ atLocation
setLocation : (Nat x Nat) -> t -> QuadTree t -> QuadTree t
setLocation = set ∘ atLocation
mapLocation : (Nat x Nat) -> (t -> t) -> QuadTree t -> t
mapLocation = over ∘ atLocation
\end{minted}
Finally,  \verb|makeTree| makes a new QuadTree with the same value everywhere, simply by calling the QuadTree constructor
\begin{minted}{agda}
makeTree : (size : Nat × Nat) -> (v : t) -> QuadTree t
\end{minted}

Additionally, QuadTree implements Functor and Foldable. The implementation of functor in the original library breaks one of the QuadTree invariants, so the implementation given in this paper was changed slightly in this implementation to not break this invariant.

\label{total-functions}
\subsection{Issues when converting Haskell to Agda}
\subsubsection{Totality}
A number of issues arise when converting Haskell to Agda, because Agda is a total language and Haskell is not. Being a total language means that every function must terminate. 

The first issue is encountered when a function in the library is actually non-terminating. This would have to be solved by changing the function, or adding preconditions, such that the function does always terminate. Luckily, all the functions in the QuadTree library do always terminate, so this was not a problem for this paper.

Additionally, the totality of Agda can also be encountered when Agda is not able to automatically prove that a function terminates, even though it does. This did actually occur during the implementation of the QuadTree library in Agda. It was initially solved by adding the \verb|\{-\# TERMINATING \#-\}| pragma (which instructs the compiler that this function is terminating) in front of the function, together with an explanation for why the function is definitely terminating. Later on this was solved by expressing the function differently.

\subsubsection{Error}
Finally, while Haskell has some escape latches such as \verb|error|, Agda does not. For example, the \verb|get/set/mapLocation| functions call \verb|error| when the provided location is outside of the QuadTree. This can be solved by adding a precondition to the function, which states that the location must be inside the QuadTree. 

This approach is different from the technique presented in \cite[p. 16]{BREITNER2021}. The technique presented there is making sure that the \verb|error| function may only be used with non-empty types, thus ensuring that the language stays sound. The technique was chosen because it allows for simpler automatic translation from Haskell. Since the translation in this paper is fully manual anyways, this does not add any value. The technique does come with the downside that the function can still be called with an incorrect input in Agda, resulting in an error at runtime rather than compile-time. Adding a precondition to the function prevents this, therefore this has been chosen.

\subsection{Necessary modifications of Agda2hs}
In order to make a working implementation, Agda2hs needed some changes.
\begin{itemize}
    \item Add support for type synonyms (Fixed in \textcolor{blue}{\href{https://github.com/agda/agda2hs/pull/56/}{PR \#56}})
    \item Insert parentheses where required in infix applications (Fixed in  \textcolor{blue}{\href{https://github.com/agda/agda2hs/pull/57/}{PR \#57}})
    \item Support for constructors with implicit arguments (Fixed in  \textcolor{blue}{\href{https://github.com/agda/agda2hs/pull/60/}{PR \#60}})
    \item Instance arguments fail to compile (Fixed in  \textcolor{blue}{\href{https://github.com/agda/agda2hs/pull/66/}{PR \#66}})
    \item Pattern matching on natural numbers does not compile correctly. (Fixed in custom version)
\end{itemize}
The first 4 problems have been solved by the Agda2hs contributors in the official Agda2hs version. The last has not been fixed in the official version, but has been fixed in a custom version which is available at \textcolor{blue}{\href{https://github.com/JonathanBrouwer/agda2hs}{https://github.com/JonathanBrouwer/agda2hs}}. Since this makes some breaking changes to Agda2hs, this has not been submitted as a PR.